\section{Modifications}
In this sectuion we introduce some modifications for basic protocol. Мы рассмотрим преобразование нашего Gamechannel в вариант без участника дилер, где будет только два равноправных игрока. В другой модификации мы рассмотрим возможность подключения третьего лица, которое не будет непосредственно участвовать в процессе игры, но будет иметь опубликованные состояния канала с подписями участниками. 
	\subsection{Two Players Case}
В некоторых играх оба игрока выступают на равных правах, а казино никак не участвует в этом. Примером могут служить некоторые разновидности дайса. В оригинальном варианте протокола, представленном  в разделе \nameref{gamechannel}, один из участников должен обеспечивать более стабильное соединение, чем другой, т.к. он ответственнен за генерацию случайного числа. Это также дает возможность для Player навязать дилеру еще один дополнительный раунд до закрытия канала. Для того, чтобы уравнять участников канала в правах мы предлагаем вариант протокола, основанный на \textit {Treshold Signature Scheme}[][]. Очевидно, treshold signature также должна обладать property of uniqueness. В качестве такой signature мы предлагаем использовать TBLS[].

Переобозначим роль игрока как $Player№1$, а роль дилера как $Player№2$.
В случае двух игроков для открытия канала мы используем протокол \autoref{alg:openchannel} с некоторыми изменениями. Подпись RSA соответственно меняется на выбранную treshold signature scheme $ \tau $ с алгоритмом $\tau .PartSign()$. This algorithm takes as input a message m and outputs participant's partial signature of the message. Второй пункт заменяется на протокол DKG, в результате которого у каждого будет часть секретного ключа и общий публичный ключ. Статус канала изменяется на $Open$, как и в оригинальном протоколе, когда контракт получает две валидные подписи от участников. 


После открытия канала раунд взаимодействий между равноправными сторонами будет происходить по новому протоколу на странице  \pageref{intchannel1} . Очевидно, после прохождения одного раунда, каждый будет иметь на руках свой подписанный результат игры и подписанное сообщение от другого участника с точно такими же данными в нем. При несовпадении данных может быть открыт диспут.
\begin{algorithm} 
\floatname{algorithm}{Protocol}
\caption*{\textbf{Protocol 2.1} Messaging in the channel}
\begin{enumerate}
	\item Участники формируют сообщения следующего вида: \label{intchannel1}
\begin{center}
$ seed\_message = (channelId, bet, round, gameData, seed)$
$signed\_seed\_message = ECDSA.Sign(seed\_message)$ 
\end{center}
 и посылают их друг другу.
	\item Участники  проверяют полученные $signed\_seed\_message$ и производят следующие вычисления:
 \begin{algorithmic}
\State $aggregate\_seed\_message =$ \\ $= seed\_message \  \text{(from Player1)} \ ||  \ seed\_message \  \text{(from Player2)}$
\State $V = H(aggregate\_seed\_message)$
\State $S =  \tau .PartSign(V)$
 \end{algorithmic}
\item Далее они обмениваются сообщениями со своими частями подписи.
\begin{center}
 $message = (S, round, gameData, player1Balance, player2Balance)$
\end{center}
	\item Игроки проверяет, что число S было рассчитано верно. Если это условие выполняется, то переходим к следующему пункту.
	\item Игроки, имея на руках две части подписи, объединяют их в одну $aggregate\_S$ в зависимости от выбранных $\tau$ и протокола DKG и рассчитывают результаты игры.
\begin{algorithmic}
\State $S_{hash} = H(aggregate\_S)$
\State $gameRange = maxGame -  minGame + 1$
\While {$S_{hash} \geq \left\lfloor 2^{hash.size} / gameRange \right\rfloor \cdot gameRange$}
\State$ S_{hash}\gets H(aggregate\_S_{hash})$
\EndWhile
\State $L = (S_{hash}$ mod $gameRange) + minGame$
\end{algorithmic}
\end{enumerate}
\end{algorithm}
\begin{algorithm}
\begin{enumerate}
\setcounter{enumi}{5}
 \item Игроки обмениваются полученными результатами игры и проверяют их. 
\begin{center}
 $message = (channelId, round, gameData, player1Balance, player2Balance)$
 $signed\_message = ECDSA.Sign(message)$
\end{center}
\item При желании один из игроков может обновить состояние канала, используя свою подпись и подпись оппонента сообщения с результатами игры.
\end{enumerate}
\end{algorithm}

Для завершения канала имеют место те же ситуации, которые были описаны в \autoref{closing}. Протоколы в этом разделе подходят для использования в данной модификации с минимальными изменениями. Теперь ставка считается сделанной, когда оба участника опубликовали свои seeds. Также функции $resolveDispute$ и $doubleSign$ доступны любому из участников. (См. пункт \ref{block} протокола 6). 

	\subsection{Third Party Observer}
Pisa, упомянутое ранее, позволяет подключить к каналу третьего наблюдателя, который, в случае потери соединения одним из пользователей, сможет представлять его интересы перед смарт-контрактом. Game Channels построены таким образом, что к ним без проблем можно применить схожую идею и подключить третью сторону. Это достигается за счет того, что каждый раз, когда участники соглашаются с каким-то состоянием, они публикуют его со своей подписью. Третья сторона прослушивает канал и затем может обновить состояние на смарт-контракте при помощи этих сообщений. Отметим, что для этого третью сторону не обязательно как-то фиксировать или верифицировать на смарт-контракте. 

Например, это может быть полезно для создания платформы, которая сводит между собой игроков и дилеров. В таком случае платформа может взять на себя ответственность за все обращения к контрактам, снизив затраты игроков на дополнительные транзакции. Обратной стороной такого подхода будет повышенная централизация системы, впрочем обращение к смарт-контракту через платформу можно сделать опциональным, но не обязательным. 
 