\documentclass[tikz, 12pt]{article}

\usepackage[margin=1.5in]{geometry}
\usepackage[table]{xcolor}
\usepackage{hyperref}
\usepackage{tikz}
\usepackage{graphicx}
\usepackage{amsmath}
\usepackage{caption}
\usepackage{amssymb}
\captionsetup{justification=raggedright,singlelinecheck=false,format=hang}
\usepackage[T2A]{fontenc}
\usepackage[utf8]{inputenc} 
\usepackage{blindtext}
\usepackage{authblk}
\usepackage{listings}
\usepackage{blindtext}
\usepackage{amsthm}
\usepackage{algorithm}
\usepackage{algpseudocode}
\usepackage{lipsum}
\newcommand*\samethanks[1][\value{footnote}]{\footnotemark[#1]}
\renewcommand{\labelenumii}{\theenumii}
\renewcommand{\theenumii}{\theenumi.\arabic{enumii}.}
\usetikzlibrary{positioning,fit,calc,arrows}
\tikzset{block/.style={draw,thick,text width=2cm,minimum height=1cm,align=center},
         line/.style={-latex}
}
\theoremstyle{definition}
\newtheorem{defn}{Definition}
\theoremstyle{remark}
\newtheorem*{remark}{Remark}
\title {Gamechannel}
\author {Alisa Chernyaeva\thanks{Maria}}
\author{Ilia Shirobokov\samethanks} 
\author{  Alexander Davydov\samethanks}
%\author[add1]{Ilia Shirobokov}
\affil{Research department, DAO.Casino Company}
\affil{ \href{mailto:Research@Dao.casino}{Research@Dao.casino}}
%\affil[]{youngPussy@Dao.casino}
%\address[add1]{Research department, company DaoCasino}
\date {August 2018}
\hyphenation{every-where}
\begin{document}
\maketitle
	\begin{abstract}
Одно из решений для масштабирования блокчейн-проектов - технология Statechannels, которая уменьшает объем транзакций и ускоряет время обмена ими в блокчейне.
Мы разработали свой вариант реализации State Channels и назвали его Game Channels. Это специализированное решение для гемблинга, оно позволяет играть двум участникам, а также участвовать третьему в качестве доверенной стороны. 
Технология Game Channels обладает большим потенциалом, т.к. ее можно использовать для других игр, не ограничиваясь только азартными.
	\end{abstract}
\tableofcontents	
	\input{sections/Introduction}
	\section{Preliminaries}
In this section, we introduce definitions and notations that will be used throughout the paper.

We denote the set of integers modulo an integer $n$ by $\mathbb{Z}_n$. When writing $x \xleftarrow{\text{R}} S$ we mean that x is chosen uniformly at random from the set $S$. By $H$ denote some cryptographic hash function. 

The communication model considered a point-to-point channels between two parties. One of them may be malicious adversary. The adversary can divert from the specified protocol in any way. 
		\section{Game Channels} \label{gamechannel}
In this section we are going to give a detailed coverage of protocols that allow two parties to open a game channel, play a game, close the channel and get rewards facing no risk of counterparty fraud. Also, we are going to consider the dispute resolution mechanism.

\begin{table}[h]

\caption{The names of variables and their meanings}
% \rowcolors{1}{lightgray}{white}
\begin{tabular}{|l|c|l|}
\hline
Name&Type&Descriprion\\
\hline
channelId & bytes32 & Unique channel identifier\\ 

playerAddress & address & Player's ethereum-address\\            
dealerAddress & address & Dealer's ethereum-address\\  
gameContractAddress & address & Ethereum-address of the game\\             
playerBalance & uint256 & Player's deposit value\\                   
dealerBalance & uint256 & Dealer's deposit value\\                   
timestamp & uint256 &  Information identifying when a message sent\\                  
RSAfingerprint & bytes32 &  Merkle-tree fingerprint of the RSA public key\\  
gameData & bytes & Game process data\\
round & uint256 & Round number of the game session \\
bet & uint256 & Player's bet \\
seed & bytes32 & Random seed for PRNG\\
totalBet & uint256 & Total amount of  player bets \\
flag & bool &Closing flag \\
maxGame, minGame & uint256 & Boundaries of random numbers in the game\\
\hline
\end{tabular}
\end{table}

\subsection {Opening a channel}
Player initiates the channel open event. To open a channel, its participants have to agree upon a specific initial state and confirm it with their signatures. Then the transaction with a state and participant signatures is sent to the smart-contract that verifies data validity. The smart-contract calculates a unique channel ID generated according to the following formula:
$channelId = H(playerAddress,  dealerAddress, timestamp)$. Note that either participant may send the opening transaction. For simplicity, let’s assume that it is sent by Dealer. The message exchange sequence is specified in the  \autoref {alg:openchannel} protocol.

\begin{algorithm}
\floatname{algorithm}{Protocol}
\caption{Opening a channel} \label{alg:openchannel}
\begin{enumerate}
	\item Player sends message containing amount of tokens for Player's deposit to Dealer.
\begin{center}
	 $initial\_message\ = (playerAddress, playerBalance, dealerAddress,$\\$ gameContractAddress)$
\end{center}
	\item Dealer generates the public RSA key $RSA\_public\_key= (N,e)$ and calculates the $RSAfingerprint$. Then, Dealer generates the following messages:
\begin{center}
	 $open\_message = (playerAddress,  dealerAddress, playerBalance, dealerBalance,$\\$timestamp, gameData, RSAfingerprint, gameContractAddress)$
	$dealer\_signed\_message = ECDSA.Sign(open\_message)$
\end{center}
	\item Dealer sends the following data to Player:
\begin{center}
$(RSA\_public\_key, open\_message,dealer\_signed\_message)$
\end{center}
	\item Player receives the message, checks data in it and then signs the $open\_message$ and sends it back to Dealer.
\begin{center}
	$player\_signed\_message = ECDSA.Sign(open\_message)$
\end{center}
	\item If the $player\_signed\_message$ is valid, Dealer calls the $openChannel$ smart contract function with the following data:
\begin{center}
$(open\_message,dealer\_signed\_message,player\_signed\_message)$
\end{center}
\begin{lstlisting}
    function openChannel(
        playerAddress,
        dealerAddress,
        playerBalance,
        dealerBalance,
        timestamp,
        gameData,
        RSAfingerprint,
        gameContractAddress,
        player_signed_message,
        dealer_signed_message
    )
\end{lstlisting}
\end{enumerate}
\end{algorithm}
\begin{algorithm}
\begin{enumerate}
\setcounter{enumi}{5}
\item The contract verifies validity of the received data. If valid, it is assumed that the both parties approved channel opening. Then the contract generates $channelId$ and freezes funds of the both parties for the game. The channel status changes to $Open$. 
\end{enumerate}
\end{algorithm}

\theoremstyle{definition}
\newtheorem{exmp}{Example}[section]
\begin{exmp}
Let’s assume that Bob runs a casino. Alice wants to use Bob’s service to play roulette available in the list of games. First Alice allows the game contract to transfer100 of Alice's tokens to later use them as a deposit. Then she sends the following message to Bob:
\\
\\
\begin{tabular}{ccc}
   	\begin{tabular}{c}
   	Alice:\\
	$0xde8456...$\\
   	\end{tabular} &
   \begin{tabular}{l}
   $\underrightarrow{\quad (0xde8456..., 100,
 0x87ff5a...) \quad}$
   \end{tabular}&
  	 \begin{tabular}{c}
 	 Bob:\\
	$0x87ff5a...$\\
   	\end{tabular} \\
\end{tabular}\\

Bob analyzes the message and agrees to carry out a game. Bob allows the game contract to transfer 5000 of Bob's tokens to later use them as a deposit. Then he responds to Alice:
\\
\\
\begin{tabular}{ccc}
   	\begin{tabular}{c}
   	Alice:\\
	$0xde8456...$\\
   	\end{tabular} &
   \begin{tabular}{c}
   $\underleftarrow{\quad (0xde8456...,  100, 0x87ff5a..., \quad $ \\  $ \quad 5000,  gameData,  timestamp,\quad $ \\  $ \quad RSAfingerprint, Bob\_signature ) \quad}$
   \end{tabular}&
  	 \begin{tabular}{cl}
 	 Bob:\\
	$0x87ff5a...$\\
   	\end{tabular} \\
\end{tabular}\\
\\

Alice checks the Bob’s message and, making sure it is valid, signs it on her part and sends back.
\\\
\\
\begin{tabular}{ccc}
   	\begin{tabular}{c}
   	Alice:\\
	$0xde8456...$\\
   	\end{tabular} &
   \begin{tabular}{c}
   $\underrightarrow{\quad (channelId,0xde8456...,  100, 0x87ff5a..., \quad $ \\  $ \quad 5000,  gameData,  timestamp,\quad $ \\  $ \quad RSAfingerprint, Alice\_signature ) \quad}$
   \end{tabular}&
  	 \begin{tabular}{cl}
 	 Bob:\\
	$0x87ff5a...$\\
   	\end{tabular} \\
\end{tabular}\\
\\

Checking Alice’s signature for validity, Bob sends both their signatures to the contract alongside the data about the channel.\\
\begin{tabular}{ccc}
   	\begin{tabular}{c}
 	 Bob:\\
	$0x87ff5a...$\\
   	\end{tabular} &
   \begin{tabular}{c}
   $\underrightarrow{\quad (channelId,0xde8456...,  100, 0x87ff5a..., \quad $ \\  $ \quad 5000,  gameData,  timestamp,\quad $ \\  $ \quad RSAfingerprint, Bob\_signature, \quad $ \\  $ \quad Alice\_signature ) \quad}$
   \end{tabular}&
  	 \begin{tabular}{cl}
 	 Contract:\\
	$0x8a4654...$\\
   	\end{tabular} \\
\end{tabular}\\
\\

The smart contract checks validity of the both signatures and the balances of the participants. Then the contract locks the parties' deposits. From this moment the channel is open.
\end{exmp}

\subsection {Interaction within the channel}




\section{Modifications}
In this sectuion we introduce some modifications for basic protocol. Мы рассмотрим преобразование нашего Gamechannel в вариант без участника дилер, где будет только два равноправных игрока. В другой модификации мы рассмотрим возможность подключения третьего лица, которое не будет непосредственно участвовать в процессе игры, но будет иметь опубликованные состояния канала с подписями участниками. 
	\subsection{Two Players Case}
В некоторых играх оба игрока выступают на равных правах, а казино никак не участвует в этом. Примером могут служить некоторые разновидности дайса. В оригинальном варианте протокола, представленном  в разделе \nameref{gamechannel}, один из участников должен обеспечивать более стабильное соединение, чем другой, т.к. он ответственнен за генерацию случайного числа. Это также дает возможность для Player навязать дилеру еще один дополнительный раунд до закрытия канала. Для того, чтобы уравнять участников канала в правах мы предлагаем вариант протокола, основанный на \textit {Treshold Signature Scheme}[][]. Очевидно, treshold signature также должна обладать property of uniqueness. В качестве такой signature мы предлагаем использовать TBLS[].

Переобозначим роль игрока как $Player№1$, а роль дилера как $Player№2$.
В случае двух игроков для открытия канала мы используем протокол \autoref{alg:openchannel} с некоторыми изменениями. Подпись RSA соответственно меняется на выбранную treshold signature scheme $ \tau $ с алгоритмом $\tau .PartSign()$. This algorithm takes as input a message m and outputs participant's partial signature of the message. Второй пункт заменяется на протокол DKG, в результате которого у каждого будет часть секретного ключа и общий публичный ключ. Статус канала изменяется на $Open$, как и в оригинальном протоколе, когда контракт получает две валидные подписи от участников. 


После открытия канала раунд взаимодействий между равноправными сторонами будет происходить по новому протоколу на странице  \pageref{intchannel1} . Очевидно, после прохождения одного раунда, каждый будет иметь на руках свой подписанный результат игры и подписанное сообщение от другого участника с точно такими же данными в нем. При несовпадении данных может быть открыт диспут.
\begin{algorithm} 
\floatname{algorithm}{Protocol}
\caption*{\textbf{Protocol 2.1} Messaging in the channel}
\begin{enumerate}
	\item Участники формируют сообщения следующего вида: \label{intchannel1}
\begin{center}
$ seed\_message = (channelId, bet, round, gameData, seed)$
$signed\_seed\_message = ECDSA.Sign(seed\_message)$ 
\end{center}
 и посылают их друг другу.
	\item Участники  проверяют полученные $signed\_seed\_message$ и производят следующие вычисления:
 \begin{algorithmic}
\State $aggregate\_seed\_message =$ \\ $= seed\_message \  \text{(from Player1)} \ ||  \ seed\_message \  \text{(from Player2)}$
\State $V = H(aggregate\_seed\_message)$
\State $S =  \tau .PartSign(V)$
 \end{algorithmic}
\item Далее они обмениваются сообщениями со своими частями подписи.
\begin{center}
 $message = (S, round, gameData, player1Balance, player2Balance)$
\end{center}
	\item Игроки проверяет, что число S было рассчитано верно. Если это условие выполняется, то переходим к следующему пункту.
	\item Игроки, имея на руках две части подписи, объединяют их в одну $aggregate\_S$ в зависимости от выбранных $\tau$ и протокола DKG и рассчитывают результаты игры.
\begin{algorithmic}
\State $S_{hash} = H(aggregate\_S)$
\State $gameRange = maxGame -  minGame + 1$
\While {$S_{hash} \geq \left\lfloor 2^{hash.size} / gameRange \right\rfloor \cdot gameRange$}
\State$ S_{hash}\gets H(aggregate\_S_{hash})$
\EndWhile
\State $L = (S_{hash}$ mod $gameRange) + minGame$
\end{algorithmic}
\end{enumerate}
\end{algorithm}
\begin{algorithm}
\begin{enumerate}
\setcounter{enumi}{5}
 \item Игроки обмениваются полученными результатами игры и проверяют их. 
\begin{center}
 $message = (channelId, round, gameData, player1Balance, player2Balance)$
 $signed\_message = ECDSA.Sign(message)$
\end{center}
\item При желании один из игроков может обновить состояние канала, используя свою подпись и подпись оппонента сообщения с результатами игры.
\end{enumerate}
\end{algorithm}

Для завершения канала имеют место те же ситуации, которые были описаны в \autoref{closing}. Протоколы в этом разделе подходят для использования в данной модификации с минимальными изменениями. Теперь ставка считается сделанной, когда оба участника опубликовали свои seeds. Также функции $resolveDispute$ и $doubleSign$ доступны любому из участников. (См. пункт \ref{block} протокола 6). 

	\subsection{Third Party Observer}
Pisa, упомянутое ранее, позволяет подключить к каналу третьего наблюдателя, который, в случае потери соединения одним из пользователей, сможет представлять его интересы перед смарт-контрактом. Game Channels построены таким образом, что к ним без проблем можно применить схожую идею и подключить третью сторону. Это достигается за счет того, что каждый раз, когда участники соглашаются с каким-то состоянием, они публикуют его со своей подписью. Третья сторона прослушивает канал и затем может обновить состояние на смарт-контракте при помощи этих сообщений. Отметим, что для этого третью сторону не обязательно как-то фиксировать или верифицировать на смарт-контракте. 

Например, это может быть полезно для создания платформы, которая сводит между собой игроков и дилеров. В таком случае платформа может взять на себя ответственность за все обращения к контрактам, снизив затраты игроков на дополнительные транзакции. Обратной стороной такого подхода будет повышенная централизация системы, впрочем обращение к смарт-контракту через платформу можно сделать опциональным, но не обязательным. 
 
	\begin{thebibliography}{9}
\bibitem{sF90}
Ran Canetti, \emph{Universally Composable Security: A New Paradigm for Cryptographic Protocols}, Boston University and Tel Aviv University, July 16, 2013.
\end{thebibliography}


		

\end{document}