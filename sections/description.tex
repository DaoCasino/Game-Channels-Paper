 	\section{Game Channels} \label{gamechannel}
В этом разделе мы подробно опишем протоколы, которые позволяют двум сторонам открыть game channel, сыграть в некоторую игру, закрыть канал и получить свои деньги, не опасаясь обмана со стороны conterparty. Также здесь мы представим механизм разрешения диспутов.\\

\begin{table}[h]

\caption{The names of variables and their meanings}
% \rowcolors{1}{lightgray}{white}
\begin{tabular}{|l|c|l|}
\hline
Name&Type&Descriprion\\
\hline
channelId & bytes32 & Unique channel identifier\\ 

playerAddress & address & Player's ethereum-address\\            
dealerAddress & address & Dealer's ethereum-address\\  
gameContractAddress & address & Ethereum-address of the game\\             
playerBalance & uint256 & Player's deposit value\\                   
dealerBalance & uint256 & Dealer's deposit value\\                   
openingBlock & uint256 &  Information identifying when a message sent\\                  
RSAfingerprint & bytes32 &  Merkle-tree fingerprint of the RSA public key\\  
gameData & bytes & Информация о ходе игры\\
round & uint256 & Round number of the game session \\
bet & uint256 & Player's bet \\
seed & bytes32 & Random seed for PRNG\\
totalBet & uint256 & Общая сумма ставок игрока \\
flag & bool &Closing flag \\
maxNumber, minNumber & uint256 & Boundaries of random numbers in the game\\
\hline
\end{tabular}
\end{table}

		\subsection {Opening a channel}
Игрок выступает в качестве инициатора открытия канала. Для открытия канала участники согласуют некоторое начальное состояние и подтверждают его своими подписями. Далее транзакцию с состоянием и подписями участников отправляют на смарт-контракт, который проверяет валидность данных. В транзакции информация о канале закрепляется за уникальным идентификатором канала $channelId = H(gameContractAddress, playerAddress, dealerAddress, playerBalance, dealerBalance, openingBlock, RSAfingerprint)$. Отметим, что транзакцию с открытием канала может отправить любой из участников. Для простоты будем считать, что эту транзакцию отправляет дилер. Последовательность выполнения обмена сообщениями отображена в протоколе \autoref {alg:openchannel}
\begin{algorithm}
\floatname{algorithm}{Protocol}
\caption{Opening a channel} \label{alg:openchannel}
\begin{enumerate}
	\item Игрок отправляет дилеру сообщение о том, что он готов сыграть на некоторую сумму токенов, которая есть в наличии на его кошельке.
\begin{center}
	 $open\_message\_player = (playerAddress, playerBalance, dealerAddress,$\\$ gameContractAddress)$
\end{center}
	\item Дилер генерирует модуль и публичный ключ RSA $RSA\_public\_key= (N,e)$ и вычисляет $RSAfingerprint$. Затем он формирует сообщение для открытия канала:
\begin{center}
	 $open\_message = (playerAddress,  playerBalance, dealerAddress,$\\$dealerBalance, gameData, openingBlock, RSAfingerprint,$\\$gameContractAddress)$
	$dealer\_signed\_message = ECDSA.Sign(open\_message)$
\end{center}
	\item Дилер отправляет игроку данные:
\begin{center}
$(RSA\_public\_key, open\_message,dealer\_signed\_message)$
\end{center}
	\item Игрок получает сообщение, проверяет помещенные в него данные, и затем сам подписывает $open\_message$ и отправляет подпись дилеру.
\begin{center}
	$player\_signed\_message = ECDSA.Sign(open\_message)$
\end{center}
	\item Если $player\_signed\_message$ валидно, то дилер отправляет следующие данные на контракт:
\begin{center}
$(open\_message,dealer\_signed\_message,player\_signed\_message)$
\end{center}
\end{enumerate}
\end{algorithm}
\begin{algorithm}
\begin{enumerate}
\setcounter{enumi}{5}
\item Контракт проверяет полученные подписи на валидность. Если они верные, то таким образом определено согласие обеих сторон на открытие канала. Затем контракт замораживает средства обоих участников для игры и изменяет статус канала на $Open$. 
\begin{lstlisting}
    function openChannel(
        playerAddress,
        dealerAddress,
        playerBalance,
        dealerBalance,
        openingBlock,
        gameData,
        RSAfingerprint,
        player_signed_message,
        dealer_signed_message
    )
\end{lstlisting}
\end{enumerate}
\end{algorithm}



\theoremstyle{definition}
\newtheorem{exmp}{Example}[section]
\begin{exmp}
Боб является владельцем казино. Алиса решает воспользоваться услугами Боба и сыграть в рулетку, представленную в списке игр казино. Для начала Алиса approve контракту игры 100 токенов, которые в дальнейшем собирается использовать в качестве депозита. Далее она посылает сообщение Бобу:
\\
\\
\begin{tabular}{ccc}
   	\begin{tabular}{c}
   	Alice:\\
	$0xde8456...$\\
   	\end{tabular} &
   \begin{tabular}{l}
   $\underrightarrow{\quad (0xde8456..., 100,
 0x87ff5a...) \quad}$
   \end{tabular}&
  	 \begin{tabular}{c}
 	 Bob:\\
	$0x87ff5a...$\\
   	\end{tabular} \\
\end{tabular}\\

Боб рассматривает сообщение Алисы, и принимает решение, что он согласен провести игру. Он посылает Алисе сообщение:
\\
\\
\begin{tabular}{ccc}
   	\begin{tabular}{c}
   	Alice:\\
	$0xde8456...$\\
   	\end{tabular} &
   \begin{tabular}{c}
   $\underleftarrow{\quad (0xde8456...,  100, 0x87ff5a..., \quad $ \\  $ \quad 5000,  gameData,  openingBlock,\quad $ \\  $ \quad RSAfingerprint, Bob\_signature ) \quad}$
   \end{tabular}&
  	 \begin{tabular}{cl}
 	 Bob:\\
	$0x87ff5a...$\\
   	\end{tabular} \\
\end{tabular}\\
\\

Алиса проверяет полученное от Боба сообщение, и, убедившись, что все верно, сама подписывает сообщение от Боба и отправляет его.
\\\
\\
\begin{tabular}{ccc}
   	\begin{tabular}{c}
   	Alice:\\
	$0xde8456...$\\
   	\end{tabular} &
   \begin{tabular}{c}
   $\underrightarrow{\quad (0xde8456...,  100, 0x87ff5a..., \quad $ \\  $ \quad 5000,  gameData,  openingBlock,\quad $ \\  $ \quad RSAfingerprint, Alice\_signature ) \quad}$
   \end{tabular}&
  	 \begin{tabular}{cl}
 	 Bob:\\
	$0x87ff5a...$\\
   	\end{tabular} \\
\end{tabular}\\
\\

Боб, удостоверившись в валидности подписи Алисы, отправляет ее и свою подпись на контракт вместе с сообщением.\\
\begin{tabular}{ccc}
   	\begin{tabular}{c}
 	 Bob:\\
	$0x87ff5a...$\\
   	\end{tabular} &
   \begin{tabular}{c}
   $\underrightarrow{\quad (0xde8456...,  100, 0x87ff5a..., \quad $ \\  $ \quad 5000,  gameData,  openingBlock,\quad $ \\  $ \quad RSAfingerprint, Bob\_signature, \quad $ \\  $ \quad Alice\_signature ) \quad}$
   \end{tabular}&
  	 \begin{tabular}{cl}
 	 Contract:\\
	$0x8a4654...$\\
   	\end{tabular} \\
\end{tabular}\\
\\

Смарт-контракт проверяет подписи Алисы и Боба. С этого момента канал открыт.
\end{exmp}



		\subsection {Interaction within the channel}
После открытия канала весь процесс игры делится на раунды. В каждом раунде игрок принимает некоторые игровые решения (например, делает ставку) и отсылает их вместе со случайным seed дилеру. Дилер на основе этого seed вычисляет результаты игры. Если игрок согласен с честностью полученных результатов, то игра переходит к следующему раунду. Процесс взаимодействия сторон в течение одного раунда представлен в протоколе \autoref{alg:intchannel}. \\
\begin{algorithm}
\floatname{algorithm}{Protocol}
\caption{Messaging in the channel} \label{alg:intchannel}
\begin{enumerate}
	\item Игрок генерирует $seed$  и  формирует сообщения:
\begin{center}
$ seed\_message = (channelId, bet, round, gameData, seed)$
$signed\_seed\_message = ECDSA.Sign(seed\_message)$ 
\end{center}
 и посылает его дилеру. 
	\item Дилер проверяет $signed\_seed\_message$, $seed\_message$ и производит следующие вычисления:
 \begin{algorithmic}
\State $V = H(seed\_message)$
\State $S = RSA.Sign(V)$
\State $S_{hash} = H(S)$
\State $gameRange = maxNumber -  minNumber + 1$
\While {$S_{hash} \geq \left\lfloor 2^{hash.size} / gameRange \right\rfloor \cdot gameRange$}
\State$ S_{hash}\gets H(S_{hash})$
\EndWhile
\State $L = (S_{hash}$ mod $gameRange) + minNumber$
 \end{algorithmic}
 Применяя игровую логику к полученного числу, дилер просчитывает результат раунда.
\item Дилер подписывает новое состояние канала
\begin{center}

$update\_channel\_message = H(channelId, playerBalance, dealerBalance, round)$
$signed\_message = ECDSA.Sign(update\_channel\_message)$

\end{center}
\item Далее он отправляет сообщение
\begin{center}

$message = (L,S, round, gameData, playerBalance, dealerBalance, signed\_message)$

\end{center}
 игроку.
	\item Игрок проверяет, что число L, результаты игры и новый баланс участников рассчитаны верно.
	\item Игрок пересылает ему $update\_channel\_message$ вместе со своей подписью. 
\begin{center}
 $signed\_message = ECDSA.Sign(update\_channel\_message)$
\end{center}
\end{enumerate}
\end{algorithm}


\begin{algorithm}
\begin{enumerate}
\setcounter{enumi}{5}
 \item Если игрок или дилер хочет обновить состояние игры, находящееся на блокчейне, то он вызывает функцию обновления в смарт-контракте:
\begin{lstlisting}
    function updateChannel(
        channelId ,
        playerBalance,
        dealerBalance,
        totalBet,
        round,
        signed_message
    )
\end{lstlisting}
\end{enumerate}
\end{algorithm}

\begin{exmp}
Продолжая вышеобозначенный пример, Алиса делает свою ставку в 10 токенов на красное, формирует сообщение и подписывает его: 
\\
\\
\begin{tabular}{ccc}
   	\begin{tabular}{c}
   	Alice:\\
	$0xde8456...$\\
   	\end{tabular} &
   \begin{tabular}{c}
   $\underrightarrow{\quad (channelId, 10, 1, gameData, qw2ert5t,\quad $ \\  $ \quad 0x8a4654..., signature ) \quad}$
   \end{tabular}&
  	 \begin{tabular}{cl}
 	 Bob:\\
	$0x87ff5a...$\\
   	\end{tabular} \\
\end{tabular}\\
\\

Боб, получив сообщение от Алисы, начинает расчитывать исход этого раунда. Для этого он находит хэш от данных Алисы и подписывает его при помощи RSA. Получаем $0x5r43c...$ Далее Боб на основе полученного числа вычисляет случайное число, определяющее результат игры. Предположим, этим числом оказывается 14. Оно располагается на красном поле. Алиса выигрывает этот раунд.
Боб формирует сообщение для Алисы:
\\
\\
\begin{tabular}{ccc}
   	\begin{tabular}{c}
   	Alice:\\
	$0xde8456...$\\
   	\end{tabular} &
   \begin{tabular}{c}
   $\underleftarrow{\quad (14, 0x5f43c..., 1, gameData, 110, 4990) \quad}$
   \end{tabular}&
  	 \begin{tabular}{cl}
 	 Bob:\\
	$0x87ff5a...$\\
   	\end{tabular} \\
\end{tabular}\\
\\

Алиса согласна с полученными данными после их проверки. Она подписывает сообщение и посылает его Бобу.
\\
\\
\begin{tabular}{ccc}
   	\begin{tabular}{c}
   	Alice:\\
	$0xde8456...$\\
   	\end{tabular} &
   \begin{tabular}{c}
   $\underrightarrow{\quad (14, 0x5f43c..., 1, gameData, 110, 4990,\quad $ \\  $ \quad signature ) \quad}$
   \end{tabular}&
  	 \begin{tabular}{cl}
 	 Bob:\\
	$0x87ff5a...$\\
   	\end{tabular} \\
\end{tabular}\\
\\

При желании любого из участников, можно отправить подписанное сообщение на контракт и обновить хранимые в нем данные. 

Далее Боб с Алисой обмениваются сообщениями до тех пор, пока кто-то из них не решит завершить канал по какой-либо причине. 
\end{exmp}

		\subsection {Closing the channel}
Если один из участников хочет завершить канал, он делает соответствующий запрос. Это может произойти по следующим причинам: \label{closing}
\begin{enumerate}
	\item Игрок или дилер решает прекратить игру по своему желанию. Протокол \autoref{alg:close1} представляет собой последовательность действий в этом случае.
	\item У кого-то из участников нулевой баланс или не хватает токенов для следующей ставки. Если один из участников посылает запрос о закрытие канала, а второй принимает, то мы действуем согласно протоколу \autoref{alg:close1}. Иначе мы используем протокол \autoref{alg:close2}.
	\item Время жизни канала вышло. Заметим, что участники должны самостоятельно следить за временем работы канала. После окончания времени действия канала все функции смарт-контракта становятся недоступны, кроме функции $closeBYTime$. Протокол \autoref{alg:close3}.
	\item Кто-то из участников прекращает отвечать на сообщения. в таком случае мы используем \autoref{alg:close4}.
	\item  Кто-то из участников подделывает данные. При попытке загрузить неправильные данные в канал, контракт присылает сообщение об ошибке. 
Таким образом, некорректные данные не могут попасть в блокчейн, и могут храниться только у участников. Когда один из участников решает, что другой обманывает его, то следует протоколу \autoref{alg:close4}. Отметим, что смартконтракт не отличает эту ситуацию и предыдущую с точки зрения внутренней реализации.
\end{enumerate}
\begin{algorithm}
\floatname{algorithm}{Protocol}
\caption{Закрытие канала по согласию} \label{alg:close1}
\begin{enumerate}
	\item Участник, который решил завершить канал, посылает сообщение c текущим состоянием раунда игры второму участнику.
\begin{center}
$message = (channelId, playerBalance, dealerBalance, totalBet, round)$
\end{center} 
	\item Если второй участник согласен с этим состоянием, то он возвращает следующее сообщение вместе с его подписью:
\begin{center}
$close\_message = (channelId, playerBalance, dealerBalance, totalBet, round, flag,$\\$ gameContractAddress)$\\
$party2\_signed\_message = ECDSA.Sign(close\_message)$
\end{center}
	\item Первый участник проверяет полученное сообщение, и, если он согласен с его содержимым, он также подписывает $close\_message$.
\begin{center}
$party1\_signed\_message = ECDSA.Sign(close\_message)$
\end{center}
	\item Дилер, имея на руках две подписи $party1\_signed\_message$ и $party2\_signed\_message$, отправляет их вместе с $close\_message$ в контракт. \label{deal_sig}
	\item Контракт осуществляет проверку валидности полученных подписей. Затем удостоверяется в равенстве сумм замороженных на контракте токенов и токенов, которые дилер и игрок собираются вывести из игры.
\begin{lstlisting}
    function closeByConsent(
        channelId,
        playerBalance,
        dealerBalance,
        totalBet,
        round,
        party1_signed_message,
        party2_signed_message 
    )
\end{lstlisting}
\end{enumerate}
\end{algorithm}

\begin{algorithm}
\begin{enumerate}
\setcounter{enumi}{5}
	\item Если условия соблюдены, то контракт запускает функцию $closeChannel(channelId)$  и распределяет токены между участниками согласно полученным данным.
	\item Контракт удаляет канал, используя функцию $removeChannel(channelId)$. Статус канала изменяется на $Close$.
\end{enumerate}
\end{algorithm}


\begin{algorithm}
\floatname{algorithm}{Protocol}
\caption{Закрытие канала из-за недостаточного баланса} \label{alg:close2}
\begin{enumerate}
	\item Первый участник обращается к функции смарт-контракта $updateChannel$, передавая ей последнее подписанное состояние, в котором у одного из участников нулевой или недостаточный для ставки баланс.
	\item Функция $updateChannel$ обновляет хранимые данные, и помимо этого, проверяет баланс участников. Если нулевой или недостаточный баланс подтверждается, то канал закрывается и удаляется. (см. пункты 6 и 7 в протоколе \autoref{alg:close1})
\end{enumerate}
\end{algorithm}


\begin{algorithm}
\floatname{algorithm}{Protocol}
\caption{Закрытие канала по окончанию времени} \label{alg:close3}
 \begin{enumerate}
	\item Один из участников вызывает функцию $closeByTime(channelId)$. Внутри этой функции проверяется, был ли открыт диспут во время ее выполнения.
	\item Если это условие выполняется, то запускается функция $closeByDispute(channelId)$, который трактует состояние канала в пользу игрока. Т.е максимальный возможный выигрыш по игровой логике отходит игроку. 
	\item Канал закрывается и удаляется. (См. пункты 6 и 7 из протокола \autoref{alg:close1}).
\end{enumerate}
\end{algorithm}

\begin{algorithm}
\floatname{algorithm}{Protocol}
\caption{Закрытие канала по причине неответа или ложных данных} \label{alg:close4}
\begin{enumerate}
	\item Участник загружает последнее согласованное состояние в смарт-контракт, используя функцию $updateChannel$.
	\item Затем участник пытается открыть диспут, используя следующую функцию в контракте:
\begin{lstlisting}
    function openDispute(
        channelId,
        round,
        playerBalance,
        gameData,
        seed,
         playerSignature
    )
\end{lstlisting}
где $ playerSignature = ECDSA.Sign(channelId, round, \\ playerBalance, gameData, seed)$.
	\item Смарт-контракт проверяет полученные данные от участника, и если они валидны, то открывается диспут. Статус канала изменяется на $Dispute$.
	\item  Участникам дается время в $t1$ блоков, чтобы вызвать $updateChannel$, в котором необходимо загрузить валидное более новое состояние канала, либо $doubleSign$, который может вызвать только дилер. В первом случае мы переходим к шагу протокола 4.1, а во втором к 4.2.
\begin{enumerate}
	\item После проверки корректности данных, $updateChannel$ сравнивает номер раунда, с которым был открыт диспут, с номером раунда в новом состоянии. Если разница строго больше единицы, то участник, открывший диспут, лишается депозита. Канал закрывается и удаляется. (См. пункты 6 и 7 из протокола \autoref{alg:close1}). Иначе после обновления контракта диспут удаляется, а статус канала обратно возвращается к $Open$. Отметим, что неважно от кого поступит обновление канала. \label{upd}
	\item  В случае $doubleSign$ смарт-контракт проверяет, не было ли получено два разных сообщения с подписью игрока с одинаковым раундом. Если это так, то игрок лишается своего депозита.  Канал закрывается и удаляется. (См. пункты 6 и 7 из протокола \autoref{alg:close1}). \label{dbl}
\end{enumerate}
\end{enumerate}
\end{algorithm}
\begin{algorithm}
\begin{enumerate}
	\setcounter{enumi}{4}
	\item По истечению $t1$ блоков у участников открывается время в $t2$ блоков, в течение которых они могут либо обновить состояние в канале или вызвать функцию $doubleSign$, либо участник с ролью дилер может вызвать функцию разрешения диспута $resolveDispute$ из контракта. Если ничего из перечисленного не происходит за $t1+t2$ блоков, то переходим к протоколу \autoref{alg:close3}. В этом случае дальнейшие шаги не выполняются. \label{block}
\begin{lstlisting}
function resolveDispute(
        channelId ,
        N,
        E,
        S
    )
\end{lstlisting}
\begin{enumerate}
	\item  Функция $resolveDispute$ проверяет RSA public key дилера и корректность RSA подписи $S$. Если они неверны, то вызов отклоняется.
	\item Функция $resolveDispute$ вызывает функцию $runGame(channelId , playerBalance, S)$
	 \item Функция $runGame$ выполняет проверку игровой логики и распределяет баланс каждого из участников.
	\item  Канал закрывается и удаляется. (См. пункты 6 и 7 из протокола \autoref{alg:close1})
\end{enumerate}
\end{enumerate}
\end{algorithm}

\begin{exmp}
Продолжаем пример. Алиса ставит в пятом раунде ставку в 20 токенов, и отправляет сообщение с данными Бобу.
\\
\\
\begin{tabular}{ccc}
   	\begin{tabular}{c}
   	Alice:\\
	$0xde8456...$\\
   	\end{tabular} &
   \begin{tabular}{c}
   $\underrightarrow{\quad (channelId, 20, 5, gameData, qwrtty,\quad $ \\  $ \quad 0x8a4654... ) \quad}$
   \end{tabular}&
  	 \begin{tabular}{cl}
 	 Bob:\\
	$0x87ff5a...$\\
   	\end{tabular} \\
\end{tabular}\\
\\

Но в этот момент у Боба обрывается соединение и он не может ответить Алисе. Алиса не получает ответа и загружает последнее согласованное с Бобом состояние в канал, а затем открывает диспут с новым запросом. 
Контракт дает время в $t1$ блоков Бобу загрузить более новое состояние игры в канал. После прохождения $t1 - 1$ блоков Боб восстанавливает соединение и успевает загрузить новое состояние с результатами игры в канал при помощи функции $updateChannel$. Диспут стирается, и Алиса с Бобом продолжают играть.

Когда время жизни канала на раунде n стало подходить к концу, Боб решил закрыть канал. Он делает запрос к Алисе, согласна ли она закрыть канал с текущим состоянием.
\\
\\
\begin{tabular}{ccc}
   	\begin{tabular}{c}
   	Alice:\\
	$0xde8456...$\\
   	\end{tabular} &
   \begin{tabular}{c}
   $\underleftarrow{\quad (channelId, 85, 5015, 375, n) \ \ \ \  \ \  \quad}$
   \end{tabular}&
  	 \begin{tabular}{cl}
 	 Bob:\\
	$0x87ff5a...$\\
   	\end{tabular} \\
\end{tabular}\\
\\

 Алиса проверяет полученное сообщение, подписывает его и отправляет ответ с данными для завершения игры Бобу.
\\
\\
\begin{tabular}{ccc}
   	\begin{tabular}{c}
   	Alice:\\
	$0xde8456...$\\
   	\end{tabular} &
   \begin{tabular}{c}
   $\underrightarrow{\quad (channelId, 85, 5015, 375, n, true, \quad $ \\  $ \quad0x8a4654..., Alice\_signature ) \quad}$
   \end{tabular}&
  	 \begin{tabular}{cl}
 	 Bob:\\
	$0x87ff5a...$\\
   	\end{tabular} \\
\end{tabular}\\
\\

 Боб в свою очередь проверяет данные от Алисы и, удостоверившись в их валидности, подписывает данные и отправляет данные со своей подписью и подписью Алисы на контракт.
\\
\\
\begin{tabular}{ccc}
   	\begin{tabular}{c}
   	Bob:\\
	$0x87ff5a...$\\
   	\end{tabular} &
   \begin{tabular}{c}
   $\underrightarrow{\quad (channelId, 85, 5015, 375, n,  true,\quad $ \\  $ \quad0x8a4654..., Bob\_signature,\quad $ \\  $ \quad Alice\_signature ) \quad}$
   \end{tabular}&
  	 \begin{tabular}{cl}
 	 Contract:\\
	$0x8a4654...$\\
   	\end{tabular} \\
\end{tabular}\\
\\

 Условие того, что обе подписи валидны, выполнено. Поэтому контракт, проверив присланные данные, закрывает канал и удаляет его.
\end{exmp}

\begin{exmp}
Параллельно с Бобом играет другой игрок, Mallory, который хочет обмануть Боба. Между ними открыт свой канал.
Mallory делает свою ставку в два токена на красное, формирует сообщение и посылает его Бобу: 
\\
\\
\begin{tabular}{ccc}
   	\begin{tabular}{c}
   	Mallory:\\
	$0xca62a232...$\\
   	\end{tabular} &
   \begin{tabular}{c}
   $\underrightarrow{\quad (channelId, 2, 1, gameData,  vg345f, \quad $ \\  $ \quad0x8a4654... ) \quad}$
   \end{tabular}&
  	 \begin{tabular}{cl}
 	 Bob:\\
	$0x87ff5a...$\\
   	\end{tabular} \\
\end{tabular}\\
\\

Боб вычисляет результат для Mallory. Это 20, которая располагается на черном поле. Mallory проигрывает свою ставку. Боб отправляет ему сообщение с исходом раунда:
\\
\\
\begin{tabular}{ccc}
   	\begin{tabular}{c}
   	Mallory:\\
	$0xca62a232...$\\
   	\end{tabular} &
   \begin{tabular}{c}
   $\underleftarrow{\quad (20, 0x5ааа3c..., 1, gameData, 3, 502) \quad}$
   \end{tabular}&
  	 \begin{tabular}{cl}
 	 Bob:\\
	$0x87ff5a...$\\
   	\end{tabular} \\
\end{tabular}\\
\\

Mallory не нравится исход раунда. Он пытается отправить состояние с неверными данными, где решение рассчитано в его сторону, Бобу. 
\\
\\
\begin{tabular}{ccc}
   	\begin{tabular}{c}
	Mallory:\\
	$0xca62a232...$\\
   	\end{tabular} &
   \begin{tabular}{c}
   $\underrightarrow{\quad (19, 0x5аа23c..., 1, gameData, 7, 498) \quad}$
   \end{tabular}&
  	 \begin{tabular}{cl}
 	 Bob:\\
	$0x87ff5a...$\\
   	\end{tabular} \\
\end{tabular}\\
\\

Боб, проверив сообщение от Mallory, распознает некорректные данные, загружает последнее валидное состояние и открывает диспут, используя функцию $openDispute$. Контракт дает Mallory время в $t1 + t2$ блоков загрузить более новое состояние, подписанное обоими участниками. Mallory снова пытается использовать подделанное состояние, загружая его в канал с помощью функции  $updateChannel$. Смарт-контракт проверяет данные, отмечает, что они невалидны, и выдает ошибку. В итоге Mallory не загружает более нового состояния на канал за $t1$ блоков. Боб теперь запускает функцию $resolveDispute$. $resolveDispute$ в свою очередь, убедившись в валидности данных Боба, вызывает функцию $runGame$, определяющую игровую логику. $runGame$ проверяет ход игры и приходит к тому, что с Mallory надо снять 2 токена, а Бобу их надо начислить на счет. После канал закрывается и удаляется. \\
\end{exmp}